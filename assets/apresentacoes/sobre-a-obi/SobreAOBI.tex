\documentclass{beamer}
\usepackage[utf8]{inputenc}

\usetheme{CambridgeUS}
\usecolortheme{default}

%------------------------------------------------------------
%This block of code defines the information to appear in the
%Title page
\title[Sobre a OBI] %optional
{Sobre a Olimpíada Brasileira de Informática}


\author{\textbf{Coordenador}\\Daniel Saad\\
\textbf{Tutores}\\Matheus Loiola\and Caio Padilha \and Leonam Knupp}

\institute[IFB] % (optional)


\date[IFB, 2022] % (optional)
{Introdução a Programação Competitiva, junho de 2021}

\logo{\includegraphics[height=1.5cm]{IFBVertical.png}}

%End of title page configuration block
%------------------------------------------------------------



%------------------------------------------------------------
%The next block of commands puts the table of contents at the 
%beginning of each section and highlights the current section:

\AtBeginSection[]
{
  \begin{frame}
    \frametitle{Tabela de Conteúdos}
    \tableofcontents[currentsection]
  \end{frame}
}
%------------------------------------------------------------


\begin{document}

%The next statement creates the title page.
\frame{\titlepage}


%---------------------------------------------------------
%This block of code is for the table of contents after
%the title page
\begin{frame}
\frametitle{Tabela de Conteúdos}
\tableofcontents
\end{frame}
%---------------------------------------------------------


\section{Sobre a OBI}

%---------------------------------------------------------
%Changing visivility of the text
\begin{frame}
\frametitle{Significado e objetivos da OBI}
A Olimpíada Brasileira de Informática (OBI) é uma iniciativa da Sociedade Brasileira de Computação que tem por objetivos:

\begin{itemize}
    \item<1-> Estimular o interesse pela Computação e por Ciências em geral.
    \item<2-> Promover a introdução de disciplinas de raciocínio computacional e técnicas de programação de computadores nas escolas de ensino médio e fundamental. 
    \item<3-> Proporcionar novos desafios aos estudantes. 
    \item<4-> Identificar talentos e vocações em Ciência da Computação de forma a melhor instruí-los e incentivá-los a seguir carreiras nas áreas de ciência e tecnologia. 
\end{itemize}
\end{frame}

\begin{frame}
\frametitle{Modalidade}
A OBI é dividida em duas modalidades com níveis diferentes cada, porém, nossa modalidade é a Programação Sênior, que engloba os alunos que estejam cursando, pela primeira vez, o primeiro ano de um curso de graduação, no momento da prova da Fase Local da OBI.
\end{frame}


\begin{frame}
\frametitle{Fases}
A OBI possui três fases: Local, Estadual e Nacional. Em cada fase, somente serão convocados para a fase seguinte os candidatos que tiverem obtido ao menos 1/3 dos pontos da prova na fase corrente e forem os melhores classificados. Abaixo, mais algumas informações sobre as fases:

\begin{itemize}
    \item<1-> Local: A prova é realizada na instituição em que o aluno está inscrito. 
    \item<2-> Estadual: A prova é realizada na instituição em que o aluno está inscrito. 
    \item<3-> Nacional: As provas da Fase Nacional serão realizadas em sedes designadas pela organização da OBI. 
\end{itemize}
\end{frame}

\begin{frame}
\frametitle{Como participar}

As inscrições vão até dia 11 de junho e é possível preencher o formulário disponível no Discord para participar na OBI, desde que os requisitos sejam preenchidos, ou seja, desde que você esteja em seu primeiro ano de graduação.

\end{frame}

\begin{frame}
\frametitle{Resultados}
Para os problemas vistos nas provas, o competidor pode receber entre zero e 100 pontos em cada tarefa, dependendo do número de testes que produziram a resposta correta.\\
\vspace{0.6cm}
Além disso, os resultados oficiais serão divulgados, juntamente com o gabarito das provas, na página oficial da OBI: \textcolor{blue}{obi.sbc.org.br}
\end{frame}

\begin{frame}
\frametitle{Premiação}

Os melhores classificados na Fase Nacional recebem medalhas de ouro, prata e bronze, e todos os participantes recebem Certificados de Participação. Além disso, é uma boa competição para ganhar experiência e treinar para o campeonato da Sociedade Brasileira de Computação (SBC).

\end{frame}

\section{Problemas}

\begin{frame}
\frametitle{Onde praticar}

Caso queira treinar para a OBI, acesse o site: \textcolor{blue}{olimpiada.ic.unicamp.br/pratique/pu/} para ter acesso aos problemas de provas anteriores.
    
\end{frame}

\begin{frame}
\frametitle{Problema 01}

\textbf{Torneio de tênis}\vspace{0.2cm}

A prefeitura contratou um novo professor para ensinar as crianças do bairro a jogar tênis na quadra de tênis do parque municipal. O professor convidou todas as crianças do bairro interessadas em aprender a jogar tênis. Ao final do primeiro mês de aulas e treinamentos foi organizado um torneio em que cada participante disputou exatamente seis jogos. O professor vai usar o desempenho no torneio para separar as crianças em três grupos, de forma a ter grupos de treino em que os participantes tenham habilidades mais ou menos iguais, usando o seguinte critério: 

\end{frame}

\begin{frame}
\frametitle{Problema 01}

\textbf{Torneio de tênis}\vspace{0.2cm}
\begin{itemize}
    \item participantes que venceram 5 ou 6 jogos serão colocados no Grupo 1; 
    \item participantes que venceram 3 ou 4 jogos serão colocados no Grupo 2; 
    \item participantes que venceram 1 ou 2 jogos serão colocados no Grupo 3; 
    \item participantes que não venceram nenhum jogo não serão convidados a continuar com os treinamentos. 
\end{itemize}
Dada uma lista com o resultado dos jogos de um participante, escreva um programa para determinar em qual grupo ele será colocado. 
\end{frame}

\begin{frame}
\frametitle{Problema 01}
\textbf{Entrada}\vspace{0.2cm}

A entrada consiste de seis linhas, cada linha indicando o resultado de um jogo do participante. Cada linha contém um único caractere: V se o participante venceu o jogo, ou P se o jogador perdeu o jogo. Não há empates nos jogos.
\end{frame}


\begin{frame}
\frametitle{Problema 01}
\textbf{Saída}\vspace{0.2cm}

Seu programa deve produzir uma única linha na saída, contendo um único inteiro, identificando o grupo em que o participante será colocado. Se o participante não for colocado em nenhum dos três grupos seu programa deve imprimir o valor -1. 

\end{frame}

\begin{frame}
\frametitle{Problema 02}
\textbf{Baralho}\vspace{0.2cm}

Uma gráfica iniciou a produção de cartas de baralho. Cada baralho produzido deve ser um baralho completo, ou seja, deve ter exatamente 52 cartas, compreendendo quatro naipes (Copas, Espadas, Ouros e Paus), com treze cartas em cada naipe (Ás, 2, 3, 4, 5, 6, 7, 8, 9, 10, Valete, Dama e Rei).

Um robô coleta cartas produzidas pelas máquinas impressoras e cortadoras e as agrupa em conjuntos de 52 cartas, preparando o baralho para ser embalado para venda. A empresa deseja garantir que cada baralho embalado seja um baralho completo e precisa de sua ajuda.

\end{frame}

\begin{frame}
\frametitle{Problema 02}
\textbf{Baralho}\vspace{0.2cm}

Dada a lista das cartas de um baralho pronto para ser embalado, escreva um programa para verificar se há cartas faltando ou duplicadas no baralho. 
\end{frame}

\begin{frame}
\frametitle{Problema 02}
\textbf{Entrada}\vspace{0.2cm}

A primeira linha da entrada contém uma cadeia de caracteres que descreve as cartas do baralho. Cada carta é descrita usando três caracteres, no formato ddN onde dd são dois dígitos decimais (de 01, representando a carta Ás, a 13, representanto a carta Rei) e N é um caractere entre C, E, U e P, representando respectivamente os naipes Copas, Espadas, Ouros e Paus). Note que o caractere que representa o naipe Ouros é U (e não O), para não confundir com o dígito zero. 

\end{frame}

\begin{frame}
\frametitle{Problema 02}
\textbf{Saída}\vspace{0.2cm}

Seu programa deve produzir exatamente quatro linhas na saída, cada linha correspondendo aos naipes Copas, Espadas, Ouros, e Paus, nessa ordem. Para cada naipe, se o conjunto de cartas está completo (ou seja, se exatamente 13 cartas com valores de 01, 02, 03, …, 12, 13 estão presentes), seu programa deve produzir o valor 0; se o conjunto de cartas tem alguma carta duplicada, seu programa deve produzir a palavra erro; se o conjunto de cartas tem cartas faltando, seu programa deve imprimir o número de cartas que faltam.

\end{frame}

\begin{frame}
\frametitle{Problema 02}
\textbf{Restrições e pontuação}\vspace{0.2cm}

\begin{itemize}
    \item 3 $\leq$ comprimento da cadeia de caracteres na entrada $\leq$ 156 
    \item para toda carta ddN, 01 $\leq$ dd $\leq$ 13 e N é C, E, U ou P. 
    \item Para um conjunto de casos de teste valendo 20 pontos, não há cartas duplicadas, há apenas cartas faltando.
\end{itemize}

\end{frame}

\end{document}
